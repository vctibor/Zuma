\documentclass{scrreprt}

\title{ZUMA}
\subtitle{Language Specification}
\date{}
\author{}

\setlength{\parindent}{0em}
\setlength{\parskip}{3mm}

\usepackage{xcolor}

\newcommand{\zuma}{ZUMA}

\newcommand{\question}[1]{
    \color{red} \textbf{QUESTION:} \color{black} #1
}

\newcommand{\note}[1]{
    \color{blue} \textbf{NOTE:} \color{black} #1
}

\begin{document}

\pagenumbering{gobble}

\maketitle

\tableofcontents

\newpage

\pagenumbering{arabic}


\chapter{Language constructs}

\section{Datatypes and constants}

\zuma{} is strongly typed.

Following datatypes can be declared using literals:

\textbf{Bool} has one of values \texttt{true} or \texttt{false}.

\textbf{Number} is a single precision floating point, i.e. f32: \texttt{1.5464}.

\textbf{Point} is declared using two numbers inside square brackets like \texttt{[4.45,6.06]}.

\textbf{Color} can be declared using sharp followed by hexadecimal value: \texttt{\#ff00a1}. Additionally few basic colors can be declared by their name: \texttt{black}, \texttt{white}, \texttt{red}, \texttt{green}, \texttt{blue} or \texttt{yellow}.

\textbf{Text} is delimited by double quotes.

We can declare constant \textit{name} with value \textit{literal} using keyword \texttt{let}:

\texttt{let <name> = <literal>;}


\section{Arithmetic and logical operations}

\zuma{} is not intended to be general purpose programming language, therefore design goal is to \textbf{minimize} amount of supported data types, operations and language constructs.

However for generating complex structures, such as ones consisting of repeated elements (\textit{for loops}) or of parts rendered conditionally (\textit{if statements}), we have to support these language constructs. And these constructs require support for aritmetic (loops) and logical (ifs) operations.

\zuma{} understands these aritmetic operations:

\textbf{Arithmetic addition} expressed by symbol \texttt{+}.

\textbf{Subtraction} expressed by symbol \texttt{-}.

\textbf{Arithmetic multiplication} expressed by symbol \texttt{*}.

\textbf{Division} expressed by symbol \texttt{/}.

And following logical ones:

\textbf{Less than} expressed by symbol \texttt{<}.

\textbf{Greater than} expressed by symbol \texttt{>}.

\textbf{Equality} expressed by symbol \texttt{==} (double equal sign).

\textbf{Logical addition (or)} expressed by symbol \texttt{|}.

\textbf{Logical multiplication (and)} expressed by symbol \texttt{\&}.

\textbf{Negation} expressed by symbol \texttt{!}.

Additionally, left and right parenthesis \texttt{( )} can be used to group operations into higher level units and set precedence of operations.

\question{Is it possible to have NO operators precedence and rely on parentheses exclusively?}


\section{Expressions}

Expressions are delimited using semicolon.

\texttt{line start = [0,10] end = [25,50] color = \#ff00a1;}

Expressions are following constructs:

\begin{itemize}
    \item constant declaration
    \item function call
    \item scope
\end{itemize}


\section{Functions}

Function is called using its name followed by list argument assignments, like this:

\texttt{line start=[0,0] end=[15,20] color=white;}

Here \texttt{line} is name of function being invoked, \texttt{start}, \texttt{end} and \texttt{color} are arguments being passed to function. Like every expression it has to end with semicolon.

There is number of predefined functions, mainly concerned with generating graphical primitives.

Support for user defined functions is planned, but not finalized.

\question{We will want to have two kind of functions, one to calculate and \textit{return} some complex value, possibly to be used inside arithmetic or logical operation. Second kind to directly render some graphical object. We will need to distuingish these. Possibly functions to directly render graphics should instead return these (with ability to assign them to constant) and only when invoked without assignment should they be rendered.}

\section{Scopes}

Scope is delimited by \texttt{\{} and \texttt{\}}. There is list of expressions between braces. Scope is an expression.


\section{Comments}

Single line:

\texttt{// this is comment}

Part-line / multiline:

\texttt{/* multiline comment */}

Comments can be nested:

\texttt{/* /* */ */}

\texttt{/* */ */}

\texttt{/* /* */}

Anything inside comments shouldn't break compilation.

\note{Comments most probably require implementation of custom lexer, and as such aren't currently implemented.}


\chapter{Predefined functions}

\section{Line}

\section{Rectangle}

\section{Text}


\chapter{Notes}

\section{Coordinate system}

Origin point is left upper corner. \texttt{x} is vertical axis, \texttt{y} is horizontal axis.

Therefore \texttt{[0,500]} describes upper right corner, while \texttt{[500,0]} describes lower left corner.


\section{Reserved words}

Color literals: \texttt{black white red green blue yellow}

Boolean literals: \texttt{true false}

Pre-defined functions: \texttt{line rectangle text}

Constant declaration keyword: \texttt{let}


\section{Architecture}

\textbf{Parser}

\textbf{Abstract Syntax Tree}

\textbf{Evaluation}: remove comments, eval variables, ifs and for loops...

\textbf{Translation}: ZUMA IR is basically in-memory OOP model of SVG to be generated. It is result of evaluation step and used to generate SVG.

\textbf{Generate SVG} 


\section{Performance goals}

\textbf{1ms} - good

\textbf{10ms} - acceptable

\textbf{100ms} - bad

\textbf{1000ms} - unacceptable


\end{document}