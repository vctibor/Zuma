\documentclass{scrreprt}

\title{ZUMA}
\subtitle{Language Specification}
\date{}
\author{}

\begin{document}

\pagenumbering{gobble}

\maketitle

\tableofcontents

\newpage

\pagenumbering{arabic}

\chapter{Datatypes}

ZUMA is strongly typed.

Following datatypes can be created using literals:

\section{Boolean}

Boolean has one of values \texttt{true} or \texttt{false}.

\section{Number}

Number is a single precision floating point, i.e. f32: \texttt{1.5464}.

\section{Point}

Point is declared using two numbers inside square brackets like \texttt{[4.45,6.06]}.

\section{Color}

Color can be declared using sharp followed by hexadecimal value: \texttt{\#ff00a1}. Additionally few basic colors can be declared by their name: \texttt{black}, \texttt{white}, \texttt{red}, \texttt{green}, \texttt{blue} or \texttt{yellow}.

\section{Text}

\chapter{Coordinate system}

Origin point is left upper corner. \texttt{x} is vertical axis, \texttt{y} is horizontal axis.

Therefore \texttt{[0,500]} describes upper right corner, while \texttt{[500,0]} describes lower left corner.

\chapter{Language constructs}

\section{Expressions}

Expressions are delimited using semicolon.

\texttt{line start = [0,10] end = [25,50] color = \#ff00a1;}

Expressions are following constructs:

\begin{itemize}
    \item constant declaration
    \item function call
    \item scope
\end{itemize}

\section{Comments}

Single line:

\texttt{// this is comment}

Part-line / multiline:

\texttt{/* multiline comment */}

Comments can be nested:

\texttt{/* /* */ */}

\texttt{/* */ */}

\texttt{/* /* */}

Anything inside comments shouldn't break compilation.

\section{Scopes}

Scope is delimited by \texttt{\{} and \texttt{\}}. There is list of expressions between braces. Scope is an expression.

\chapter{Architecture}

\section{Parser}

\section{Abstract Syntax Tree}

\section{Evaluation}

remove comments, eval variables, ifs and for loops

\section{ZUMA IR}

\section{Translation}

ZUMA IR to SVG model

\section{Generate SVG}


\end{document}